\setcounter{section}{-1}

\begin{abstract}
    This first Section very much retreads basic topics and concepts, though still definitely worth reading. Especially since on occasion it gives deeper insights into the building blocks/backbones of later concepts! These notes won't be as in depth as later ones sorry future me :(
\end{abstract}

\section{Foundations}
General summary:
\begin{itemize}
    \item Interesting recap, a few deeper peeks at foundations of functions
    \item Discussed in more detail the concepts of; method of exhaustion, limits, original method of calculating \(\pi\), area and rate of change (problems solved by integrals and derivatives respectively)
    \item Brief overview of the applications of these concepts in various other aspects of study
\end{itemize}

\section{Functions}
\subsection{Representations}
General Summary:
\begin{itemize}
    \item Discusses various ways to represent functions and the basic core concept of functions as mappings from one set to the other as in the endnote. \endnote {
        \import{Endnotes/}{FunctionsAsSetMaps}
    }
    \item Representations include: Tables, Graphs, word form and algebraic, of which both table and algebraic can be seen in the first endnote
    \item The chapter also revisits the contexts in which separate representations of the same function may pose certain benefits or drawbacks.
\end{itemize}
\subsection{Transformations}
General Summary:
\begin{itemize}
    \item Re-introduces the process of mathematical modeling and it's steps 
    \begin{enumerate}
        \item Starting with an examination of a real world problem
        \item Then producing a mathematical model 
        \item From this model conclusions can be made
        \item Which may then be used to generate real world predictions
        \item These predictions often raise more questions which returns the process to the start
    \end{enumerate}
    \item This chapter also reminds of various common functions used in mathematical modeling, drawing on concepts from section 1.1 to discuss a variety of categories such as; linear, polynomial, trigonometric, exponential, logarithmic and more!\endnote{
        \import{Endnotes/}{Func-ex}
    }
\end{itemize}
\newpage

\theendnotes


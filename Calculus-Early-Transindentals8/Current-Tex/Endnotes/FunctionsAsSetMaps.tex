Consider the sets A and B such that
\[
A,B \subset{\mathbb{R}}
\]
From this we can infer that if \(B\) is a function of \(A\) then every point in then every element in \(B\) is mapped to one in \(A\), in other words.
\[
B = f(A) \rightarrow B_n = A_n
\]
A more practical would be the function of 
\[
 \{(x,y) \; | \; x \in \mathbb{Z} \wedge -5 \leq x \leq 10\} 
\]
Let us consider it first as the table shown below.

\begin{table}[htp]
    \centering
    \begin{tabular}{cc}
    \toprule
    x & y \\ 
    \midrule
        \(-5\) & \(25\) \\
        \(-4\) & \(16\) \\
        \(-3\) & \(9\) \\
        \(-2\) & \(4\) \\
        \(-1\) & \(1\) \\
        \(0\) & \(0\) \\
        \(1\) & \(1\) \\
        \(2\) & \(4\) \\
        \(3\) & \(9\) \\
        \(4\) & \(16\) \\
        \(5\) & \(25\)
    \\ \bottomrule\end{tabular}
    \caption{\(f(x) = x^2\) in domain \([-5, 5] \in \mathbb{Z}\)}
\end{table}

Here we can see that each element in set \(x\) has a corresponding element in y that is to say, \(x_n \rightarrow y_n\) this is as we know also the case when \(x \in \mathbb{R}\) though this definition was not used in the table for readability.

Consider how in the above table – an x value of 4 corresponds to a y value of 16 – that is to say \(x = 4 \rightarrow y=16\) or \(f(4) = 16\).


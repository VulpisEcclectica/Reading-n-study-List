Linear:
\[
 y = f(x) = mx + b
\]
Polynomials:
\[
  P(x) = a_nx^n+a_{n-1}x^{n-1}+\cdots+a_2x^2+a_1x^1+a_0
\]
where \(n\) is a non-negative integer (e.g. \(\mathbb{N}_0\)) and \(a \subset \mathbb{R} \: | \: a_n \neq 0\)
For example the traditional cubic formula is 
\[
 P(x) = ax^3+bx^2+cx +d, \;\;\;\; a \neq 0
\]
Power Functions:
\begin{equation} \label{(1)}
a = n \; | \; n \in \mathbb{N}_0 
\end{equation}
\begin{equation} \label{(2)}
a = \frac{1}{n} \; | \; n \in \mathbb{N}_0 
\end{equation}
\begin{equation} \label{(3)}
a = -1   
\end{equation}
Rational Functions:
\[
f(x)=\frac{P(x)}{Q(x)}
\]
where:
\[
\neg \exists x \;|\;Q(x) =0
\]
Trigonometric functions:
NOTE: ALWAYS USE RADIANS
\begin{equation} \label{sine}
    f(x)=\sin x
\end{equation}
\begin{equation} \label{cosine}
    f(x)=\cos x
\end{equation}
\begin{equation} \label{tan}
    f(x)=\tan x
\end{equation}
Exponential Functions:
\[
f(x) = b^x \; \{b \in \mathbb{R}\;|\; 0 < b \;\wedge \frac{db}{dx}=0\}
\]
Logarithmic functions:
\[
f(x) = \log_b x \; \{b \in \mathbb{R}\;|\; 0 < b \;\wedge \frac{db}{dx}=0\}
\]
